%% 	This LaTeX Template is meant for "Second Bachelor´s Thesis"
%% 	written by Frank A Titze in March 2009
%% 	for Hochschule Kufstein Tirol, Austria, Dep. Wirtschaftsinformatik
%% 	www.hsk-edu.at
%% 	@author Frank A Titze (frank.a.titze@gmx.eu)
%% 	@copyright 2009, FAT
%% 	@license GPL
%% 	@version 1.0
%%=================================================
%% 	Main Document  Vorlage_BA2_by.fat.tex
%% 	This document:  exsummary.tex
%%=================================================

\chapter*{Abkürzungsverzeichnis}
\addcontentsline{toc}{section}{Abkürzungsverzeichnis} 
\begin{table}[h]
	\label{my-label}
	\begin{tabular}{ll}
		UGC & User-Generated Content \\
	
	\end{tabular}
\end{table}

\chapter*{Kurzfassung}

FH Kufstein\newline
Web Business \& Technology\newline
Kurzfassung der Bachelorarbeit \textit{Analyse des Entscheidungsverhaltens von Bergsportlern bei der Risikobewertung bei der Auswahl der Route beim Ausüben von Bergsport mit Zuhilfenahme von mobilen Applikationen und Online-Inhalten}\newline
Paula Engelberg\newline
Dr. Michael Kohlegger


Den Bergsportlern stehen heutzutage zahlreiche Möglichkeiten zur Verfügung, um sich auf Wandertouren vorzubereiten. Dazu zählen Informationen  aus Interessensplattformen, anderen Online-Inhalten und mobile Applikationen. Hier finden Wanderer genaue Angaben über Distanz, Höhenmeter, Beschaffenheit der Strecke und Einkehrmöglichkeiten. Doch, trotz der vielen Möglichkeiten zur Einholung von Informationen, steigt laut der Statistik des österreichischen Kuratoriums für Alpine Sicherheit die Zahl der Alpinen Notrufe von unverletzten Personen im letzten Jahrzehnt von Jahr zu Jahr an.

Zweck dieser Bachelorarbeit ist es, herauszufinden, ob die Fülle an Online-Inhalten sowie die Zuhilfenahme von mobilen Applikationen den Wanderern ein gewisses Sicherheitsgefühl gibt, da sie glauben, dass sie nichts mehr überraschen kann und sie sich deshalb für eher schwierigere und risikoreichere Wandertouren entscheiden. Um die Forschungsfrage zu beantworten, wurde zuerst eine Literaturanalyse durchgeführt um die wichtigsten Begrifflichkeiten zu erläutern. In weiterer Folge wurden neun leitfadengestützte Experteninterviews durchgeführt. Das Resultat dieser Interviews zeigt, dass Wanderinnen und Wanderer sich durchaus durch den Konsum von Online-Inhalten und der Zuhilfenahme von mobilen Applikationen in der Risikobewertung von Wandertouren beeinflussen lassen und es kann vorkommen, dass dadurch risikoreichere Touren gewählt werden.




\chapter*{Executive Summary}

FH Kufstein\newline
Web Business \& Technology\newline
Kurzfassung der Bachelorarbeit \textit{Analysis of the decision behaviour of mountaineers in risk assessment when choosing a trail for mountain sports with the help of mobile applications and online content}\newline
Paula Engelberg\newline
Dr. Michael Kohlegger

Today, mountaineers have numerous opportunities to prepare themselves for hiking tours. This includes information from interest platforms, other online content and mobile applications. Here hikers will find precise information about the distance, altitude difference, characteristics of the trail, refreshment and accommodation stops. Despite the many possibilities for collection of information, however, according to the statistics of the Austrian Board of Trustees for Alpine Safety, the number of Alpine emergency calls from unharmed persons has been increasing from year to year over the last decade.

The purpose of this bachelor thesis is to evaluate whether the vast amount of online content and the support of mobile applications provides hikers with a certain sense of security, as they believe that nothing can surprise them any more and that they are therefore opting for more difficult and riskier hiking tours. 

In order to answer the research question, a literature analysis was performed and the most important terms have been explained. Furthermore, nine guided expert interviews have been conducted. 
The results of these interviews show that hikers can be influenced by the consumption in terms of the risk assessment of hiking tours and it can be the case that higher-risk tours are chosen.



%%EOF@fat
