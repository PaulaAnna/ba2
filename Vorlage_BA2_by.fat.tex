%% 	This LaTeX Template is meant for "Second Bachelor´s Thesis"
%% 	written by Frank A Titze in March 2009
%% 	for Hochschule Kufstein Tirol, Austria, Dep. Wirtschaftsinformatik
%% 	www.hsk-edu.at
%% 	@author Frank A Titze (frank.a.titze@gmx.eu)
%% 	@copyright 2009, FAT
%% 	@license GPL
%% 	@version 1.0
%%=================================================
%% 	Main Document  Vorlage_BA2_by.fat.tex
%% 	This document:  Vorlage_BA2_by.fat.tex
%%=================================================


% Erste Konfigutationen einbinden
%% 	This LaTeX Template is meant for "Second Bachelor´s Thesis"
%% 	written by Frank A Titze in March 2009
%% 	for Hochschule Kufstein Tirol, Austria, Dep. Wirtschaftsinformatik
%% 	www.hsk-edu.at
%% 	@author Frank A Titze (frank.a.titze@gmx.eu)
%% 	@copyright 2009, FAT
%% 	@license GPL
%% 	@version 1.0
%%=================================================
%% 	Main Document  Vorlage_BA2_by.fat.tex
%% 	This document:  config.inc.tex
%%=================================================
\documentclass[
	a4paper,
	oneside,
	11pt,
%	DIVcalc, 	% Bei Problemen mit den Seitenrändern testweise aktivieren
%	BCOR=5mm, 	% Zusätzlicher Binderand links
	halfparskip, 	% Absatzabstand statt Einrückung neuer Zeile
	headsepline, 	% Linie nach der Kopfzeile
%	bigheadings, 	% "grössere" Überschriften
	pointlessnumbers,
	bibtotoc,
	liststotoc,
	pdftex,
]{scrreprt}
\usepackage{cmap}		% Für ein durchsuchbares PDF
\usepackage[T1]{fontenc}
\usepackage{ucs} 	% UTF8 initialisieren
\usepackage[utf8x]{inputenc} 	% UTF8 benutzen
\usepackage[english,ngerman]{babel} 	% Einstellungen Deutsch, aber zuvor Englisch geladen, um einige Einstellungen auf default zu setzen
\usepackage{lmodern}
\usepackage{palatino} 	% Schriftsatz
\renewcommand{\sfdefault}{lmss} %default Sans Serif Font; lmss == lmodern Sans Serif
\usepackage{ifpdf}
\usepackage{url} 	% URLs formatieren
\usepackage{graphicx} 	% Grafikpackage
\usepackage[usenames]{color} 	% Farben
\usepackage{lipsum} 	% mit \ipsum Blindtext einfügen
\usepackage{texlogos} 	% TeX Logos
\usepackage{eurosym} 	% EUR Symbol
\usepackage{setspace} 	% Zeilenabstand
\usepackage{microtype} 	% Optischer Randausgleich
\usepackage{romannum} 	% Römische Zahlen mit \romannum oder \Romannum

\setcounter{tocdepth}{4} 	% Gliederungstiefe im Inhaltsverzeichnis; Jarz´sche Vorgabe ist "vollständig"
\setcounter{secnumdepth}{3} 	% Gliederungstiefe bis zu der Nummeriert wird

\usepackage{listings} %Für Listings (einbinden eines Quellcodes)
%Festlegungen für listings
	\definecolor{MyCodeColor}{rgb}{0.85,0.92,0.94}
\lstset{
  language=PHP, %Programmiersprache; z.B.: PHP, C, C++, HTML, Java, SQL, etc. (see "listings"-Package for more Information)
  numbers=left, %Zeilennummerierung auf der linken Seite
  numberstyle=\scriptsize, %Zeilennummerierung kleine Schriftgröße
  stringstyle=\ttfamily, 	% Courier Schriftart für Strings
  showstringspaces=false, % In Strings keine Backspace zeichen
  numbersep=5pt, % Abstand der Zeilennummern zum Code
  basicstyle=\small, %Schriftgröße Small
  breakatwhitespace=true,
  breaklines=true,%
  tabsize=4, %Tabulator wie in VisoalC++
  commentstyle=\color{green}, %Kommentarfarbe
  keywordstyle=\color{blue}\bfseries, %Keywörterfabe
  backgroundcolor=\color{MyCodeColor},
}%
% \renewcommand{\name}[Anzahl]{Definition #1}
\makeatletter
\let\NAT@parse\undefined
\newcommand{\BibTeX}{\bibtexlogo} 	% Kompatiblität
\makeatother

% Zitationen in Fußnoten erlauben
\usepackage[round,authoryear,sort]{natbib} 	% für Darstellung der Literaturverweise im Text
\newcommand{\footcite}[2][]{\footnote{\citealp[~vgl.][#1]{#2}}} 	% Zitierstellen in Fussnoten

% Herausgestellte Zitate Kursiv
\newenvironment{zitat}%
{\begin{quote}\itshape}%
{\end{quote}}%

% Itemabstände bei Listen verringert
\let\origitemize\itemize
\def\itemize{\origitemize\itemsep-4pt}
\let\origenumerate\enumerate
\def\enumerate{\origenumerate\itemsep-4pt}
\let\origdescription\description
\def\description{\origdescription\itemsep-4pt}

% Schrift in der Kopfzeile
\setkomafont{pagehead}{\scriptsize\slshape} 	% Schrift in der Kopfzeile

% Literaturverzeichnis in Quellenverzeichnis umbenennen
\addto{\captionsngerman}{%<--entsprechend der gewählten Sprache
  \renewcommand{\bibname}{Quellenverzeichnis} 	% statt "Literaturverzeichnis"
%  \renewcommand{\figurename}{Abb.} 	% Abbildung durch Abb. ersetzen
%  \renewcommand{\tablename}{Tab.} 	% Tabelle durch Tab. ersetzen
}%

% Formatierungen für Bild- und Tabellenunterschriften
\usepackage[format=plain, 	% Bildunterschriften
	justification=centerlast, 	% Die letzte Zeile wird zentriert
	labelfont={small,bf,it}, 	% Abbildung <Nr> wird small, fett und italic gesetzt
	textfont={small,singlespacing,it}, 	% Der Text wird small und italic mit einfachem Zeilenabstand gesetzt
	width={0.8\textwidth}
]{caption}

%%EOF@fat





% Hier diverse Daten angeben
\newcommand{\mytitle}{Analyse des Entscheidungsverhaltens von Bergsportlern bei der Risikobewertung bei der Auswahl der Route beim Ausüben von Bergsport mit Zuhilfenahme von mobilen Applikationen und Online-Inhalten}
\newcommand{\arbeit}{Bachelorarbeit \Romannum{2}}
\newcommand{\jahrgang}{Web Business \& Technology 2015}
\newcommand{\abgabedat}{11. Juni 2017}
\newcommand{\betreuer}{Dr. Michael Kohlegger}
\newcommand{\autor}{Paula Engelberg}
\newcommand{\personalnummer}{1510653183}
\newcommand{\mykeywords}{} 	% Meta-Keywords der Arbeit fürs PDF



%============================================================
% Weitere Konfigurationen und Abkürzungen
%============================================================
%% 	This LaTeX Template is meant for "Second Bachelor´s Thesis"
%% 	written by Frank A Titze in March 2009
%% 	for Hochschule Kufstein Tirol, Austria, Dep. Wirtschaftsinformatik
%% 	@copyright 2009 Frank A Titze
%% 	@license GPL
%%=================================================
%% 	Main Document  Vorlage_BA2_by.fat.tex
%% 	This document:  config.pdf.tex
%%=================================================

%% 	PDF und Links formatieren
\usepackage[pdftex,pagebackref,pdfa]{hyperref}
\hypersetup{plainpages=false,
	pdftitle={\mytitle},
	pdfauthor={\autor},
	pdfsubject={Bachelor of Arts in Business (BA)},
	pdfcreator={MiKTeX}, 
	pdfproducer={Template by Frank A Titze},
	pdfkeywords={\mykeywords},
	colorlinks=true,
	linkcolor=black,
	citecolor=black,
	urlcolor=black,
	a4paper,
	breaklinks=true,
%	linktocpage=true, % verlinkt nur die Seitenzahlen im tableofcontents, da bei Umbruch der Link kaputt gehen kann
	bookmarksopen=true,
	bookmarksnumbered=true,
	bookmarksopenlevel=1,
	pdfmenubar=true,
	pdfwindowui=true,
	pdfview=FitV,
	pdfstartview=Fit,
	pdffitwindow=true
}
\pdfcompresslevel=1 %% ZLib Komprimierung: 0==none; 1==fastest; 9==best
\pdfimageresolution=1200 %% Bilder in 600dpi ; Standard ist 300dpi

% Rückreferenzentext zur Literatur im Quellenverzeichnis (nach hyperref!)
\renewcommand*{\backreftwosep}{ und~}
\renewcommand*{\backreflastsep}{ und~}
\renewcommand*{\backref}[1]{}
\renewcommand*{\backrefalt}[4]{%
\ifcase #1 %
 (nicht zitiert).%
\or
 (zitiert auf Seite~#2).%
\else
 (zitiert auf den Seiten~#2).%
\fi
}

% Glossar / Abkürzungsverzeichnis / Symbolverzeichnis; nach hyperref
\usepackage[
nonumberlist, %keine Seitenzahlen anzeigen
toc,          %Eintrag im Inhaltsverzeichnis
]{glossaries}


%%EOF@fat

%% 	This LaTeX Template is meant for "Second Bachelor´s Thesis"
%% 	written by Frank A Titze in March 2009
%% 	for Hochschule Kufstein Tirol, Austria, Dep. Wirtschaftsinformatik
%% 	www.hsk-edu.at
%% 	@author Frank A Titze (frank.a.titze@gmx.eu)
%% 	@copyright 2009, FAT
%% 	@license GPL
%% 	@version 1.0
%%=================================================
%% 	Main Document  Vorlage_BA2_by.fat.tex
%% 	This document:  glossary.tex
%%=================================================

% Länge der gepunkteten Linie
\setlength{\glslistdottedwidth}{.2\linewidth}
% Den Punkt am Ende jeder Beschreibung deaktivieren
\renewcommand*{\glspostdescription}{}
% Weitere Formatierungen
\glsSetSuffixF{\nohyperpage{f.}}
\glsSetSuffixFF{\nohyperpage{ff.}}
% Kompilieranweisungen
\makeindex 	% Sortieren
\makeglossaries 	% Abkürzungsverzeichnis erstellen


% Abkürzungen
% \newacronym{VerweismarkeText}{Abkürzung}{Erklärung}
% im Text dann:  ... erreichen wir bei \gls{VerweismarkeText}-Systemen ...
\newacronym{ms}{MS}{Microsoft}
\newacronym{cd}{CD}{Compact Disc}
\newacronym{ad}{AD}{Active Directory}

%%EOF@fat
 	% Glossar, Abkürzungsverzeichnis und/oder Symbolverzeichnis

%============================================================
% Dokumentanfang und Titelei
%============================================================
\begin{document}
	%% 	This LaTeX Template is meant for "Second Bachelor´s Thesis"
%% 	written by Frank A Titze in March 2009
%% 	for Hochschule Kufstein Tirol, Austria, Dep. Wirtschaftsinformatik
%% 	www.hsk-edu.at
%% 	@author Frank A Titze (frank.a.titze@gmx.eu)
%% 	@copyright 2009, FAT
%% 	@license GPL
%% 	@version 1.0
%%=================================================
%% 	Main Document  Vorlage_BA2_by.fat.tex
%% 	This document:  title.tex
%%=================================================

%============================================================
% Äussere Titelseite (Schmutztitel)
%============================================================
\begin{titlepage} 
	\thispagestyle{empty}
	\setcounter{page}{-2}

\begin{center}
	\hspace{1cm}
\vfill

\begin{minipage}{0.99\textwidth}
\begin{center}
\onehalfspacing
\Huge \textsf{\textbf{\mytitle}}
\end{center}
\end{minipage}

\vfill
	\hspace{1cm} \\ \vspace{2em}
\vfill
	\hspace{1cm} \\ \vspace{2em}
\vfill

\textbf{\textsf{\LARGE{%
\autor \\
\arbeit \\
}}}%

\vspace{1em}
\LARGE{%
Jahrgang \jahrgang \\
}%

\end{center}

%============================================================
% Leerseite plus Innere Titelseite
%============================================================
	\newpage
	\thispagestyle{empty}
	\hspace{1cm}
	\newpage
	\thispagestyle{empty}
\begin{center}

\mbox{
		\includegraphics[width=0.3\linewidth]{../fh_logo}
}


\vfill

%{\Huge \textsf{\textbf{\mytitle}}}
\begin{minipage}{1\textwidth}
\begin{center}
\onehalfspacing
\Huge \textsf{\textbf{\mytitle}}
\end{center}
\end{minipage}

\vfill

{\LARGE \textsf{\textbf{\\ Bachelorarbeit}}}

\large
zur Erlangung des akademischen Grades \\
Bachelor of Science in Engineering (BSc.)

\vspace{1em}

Eingereicht an der \\
Hochschule Kufstein Tirol\\
Studiengang Web Business \& Technology

\vspace{1em}

Vorgelegt von\\
\autor \\
\personalnummer \\

\vspace{1em}

Gutachter \\
\betreuer 

\vspace{1.5em}

\abgabedat


\end{center}
\end{titlepage}

%%EOF@fat
 	% Einbinden der Titelei
\pagenumbering{roman}
	%% 	This LaTeX Template is meant for "Second Bachelor´s Thesis"
%% 	written by Frank A Titze in March 2009
%% 	for Hochschule Kufstein Tirol, Austria, Dep. Wirtschaftsinformatik
%% 	www.hsk-edu.at
%% 	@author Frank A Titze (frank.a.titze@gmx.eu)
%% 	@copyright 2009, FAT
%% 	@license GPL
%% 	@version 1.0
%%=================================================
%% 	Main Document  Vorlage_BA2_by.fat.tex
%% 	This document:  eidesstattliche.tex
%%=================================================

\chapter*{Eidesstattliche Erklärung}

\textsl{Ich erkläre hiermit, dass ich die vorliegende Bachelorarbeit ohne fremde Hilfe selbständig verfasst und in der Bearbeitung und Abfassung keine anderen als die angegebenen Quellen oder Hilfsmittel benutzt sowie wörtliche und sinngemäße Zitate als solche gekennzeichnet habe. Die vorliegende Bachelorarbeit wurde noch nicht anderweitig für Prüfungszwecke vorgelegt.}
\vspace{1cm}

\makebox[0.45\textwidth]{%
Wien, den \today
}%

\vspace{2cm}

\rule{0.45\textwidth}{0.3pt} \\
\makebox[0.45\textwidth]{%
{\autor}
}%

%%EOF@fat


%============================================================
% Vorspann mit kleinen römischen Seitenzahlen
%============================================================
	\tableofcontents
		\clearpage
	\listoftables 	% Tabellenverzeichnis
	\listoffigures 	% Abbildungsverzeichnis
%	\printglossary[toctitle=Abkürzungsverzeichnis,style=super,title=Abkürzungsverzeichnis] 	% Glossar, Abkürzungsverzeichnis und/oder Symbolverzeichnis ausgeben
	\printglossary[toctitle=Abkürzungsverzeichnis,style=listdotted,title=Abkürzungsverzeichnis] 	% Glossar, Abkürzungsverzeichnis und/oder Symbolverzeichnis ausgeben
\clearpage
\onehalfspacing
	%% 	This LaTeX Template is meant for "Second Bachelor´s Thesis"
%% 	written by Frank A Titze in March 2009
%% 	for Hochschule Kufstein Tirol, Austria, Dep. Wirtschaftsinformatik
%% 	www.hsk-edu.at
%% 	@author Frank A Titze (frank.a.titze@gmx.eu)
%% 	@copyright 2009, FAT
%% 	@license GPL
%% 	@version 1.0
%%=================================================
%% 	Main Document  Vorlage_BA2_by.fat.tex
%% 	This document:  exsummary.tex
%%=================================================

\chapter*{Abkürzungsverzeichnis}
\addcontentsline{toc}{section}{Abkürzungsverzeichnis} 
\begin{table}[h]
	\label{my-label}
	\begin{tabular}{ll}
		CQL & Cassandra Query Language \\
		DBMS    & Datenbank Management System          \\
		DMS   & Data Messaging System \\
		POJO & Plain Old Java Object \\
		SOAP & Simple Object Access Protocol \\
		SQL & Structured Query Language
	\end{tabular}
\end{table}

\chapter*{Kurzfassung}


\chapter*{Executive Summary}


%%EOF@fat
 	% Abstract

%============================================================
% Hauptteil
%============================================================
\clearpage
\pagestyle{headings}
\onehalfspacing
\pagenumbering{arabic}
	%% 	This LaTeX Template is meant for "Second Bachelor´s Thesis"
%% 	written by Frank A Titze in March 2009
%% 	for Hochschule Kufstein Tirol, Austria, Dep. Wirtschaftsinformatik
%% 	www.hsk-edu.at
%% 	@author Frank A Titze (frank.a.titze@gmx.eu)
%% 	@copyright 2009, FAT
%% 	@license GPL
%% 	@version 1.0
%%=================================================
%% 	Main Document  Vorlage_BA2_by.fat.tex
%% 	This document:  mainmatter.tex
%%=================================================

%% Hier soll der Hauptteil der Arbeit erscheinen
%% auskommentieren der folgenden Zeile entfern den Beispielcontent
%% \input{misc/beispiele} 	% Einbinden der Beispiele


\chapter{Einleitung}
\label{ch:Einleitung}
Dieses Kapitel gibt eine Einführung in das Thema der Bachelorarbeit. Dabei wird auf die Ausgangssituation und den Aufbau der Arbeit eingegangen. Des weiteren wird die Problemstellung der Arbeit definiert und die Relevanz des Themas beschrieben.


\section{Ausgangssituation}

Anfangs war das Internet lediglich eine Informationsquelle für den Menschen. Mitte der 2000er-Jahre trat eine Veränderung ein, Internetnutzer wollten Teil des Webs werden, sie wollten selbst Inhalte generieren und mit anderen Internetnutzern teilen. So entwickelte sich das Internet zu einem interaktiven Web, welches unter dem Begriff des \glqq Web 2.0\grqq  durch Tim O'Reilly wesentlich geprägt wurde\footcite{oreilly}.

Auch hat die Zahl der Mobiltelefone in den letzten 15 Jahren sehr stark zugenommen.
Heutzutage sind Mobiltelefone und Internetzugang aus unserem alltäglichen Leben nicht mehr wegzudenken. Aus dem Mobile Communications Report 2017 der Mobile Marketing Assosiation Austria und der MindTake Research geht hervor, dass 2017 bereits 94\% der Österreicher ein Smartphone besitzen. 93\% der Österreicher nutzen das Internet regelmäßig auf ihrem Smartphone, in der Altersklasse der 15- bis 29-Jährigen, sind es sogar 100\%. Die tägliche Nutzung liegt bei über drei Stunden pro Tag\footcite{MMAA}.

Die schnell voranschreitende Entwicklung neuer Technologien bietet Menschen sehr viele Möglichkeiten zu kommunizieren und sich mit anderen Menschen auszutauschen. Auch sind Menschen nicht mehr an das Internet zu Hause gebunden, sie können mobiles Internet auf ihren Mobiltelefonen, Tablets und Wearables unterwegs nutzen und live Informationen einholen. Die Entgeräte sind leistungsfähiger, benutzerfreundlicher, transportabel und intelligenter als je zuvor.

Bergsportlern stehen zahlreichen mobile Applikationen, Internetforen, Plattformen und Blogs für die Tourenplanung zur Verfügung. Touren können online angeschaut und geplant werden. Auf Interessensplattformen und Blogs beschreiben Bergsportler ihre eigenen Touren und berichten über ihre  Erlebnisse. Sie schreiben persönliche Reiseberichte mit vielen Details, diese können von anderen Internetnutzern gelesen und kommentiert werden. Eine Interaktion zwischen Verfasser, Leser und anderen Lesern entsteht. Diese Berichte können Bergsportlern als Motivation und Inspiration für zukünftige Touren dienen. Hier kann jedoch das Problem lauern, denn nicht jeder Bergsportler ist gleich fit und gleich erfahren.


\section{Problemstellung}

Das Angebot für Tourenplanung ist sehr groß, es reicht von gedruckten Tourenführern über Online-Plattformen zu GPS-fähigen Wearables. Bergtouren können heutzutage besser den je geplant werden, Online-Tools und mobile Applikationen geben Auskunft über die Distanz, Höhenmeter, Beschaffenheit der Strecke und Einkehrmöglichkeiten. Auch das Wetter ist jederzeit zeitnah abrufbar. Dennoch lässt sich der Presseaussendung von Januar 2018 des österreichischen Kuratoriums für alpine Sicherheit entnehmen, dass die Zahl der unverletzten Bergsportler, welche einen alpinen Notruf absetzen, in den vergangenen 10 Jahren stetig anstieg. Im Jahr 2017 machten Unverletzte bereits 37\% aller Notrufe aus. Alpine Notrufe werden nicht ausschließlich von tatsächlich Verunfallten und Verletzten abgesetzt, sondern auch von unverletzten Personen, die sich in einer misslichen Lage befinden\footcite{kurasi}.
Um im weiteren Verlauf bessere Aussagen treffen zu können, widmet sich diese Arbeit den Wanderern. Wandern ist der beliebteste Bergsport und wird von Personen jeden Alters und Geschlechts ausgeübt.
In der Statistik der letzten 10 Jahre kann man entnehmen, dass die Zahl der Unverletzten von 425 im Zeitraum 01.11.2007 bis 31.10.2008 auf 811 im Zeitraum 01.11.2015 bis 31.10.2016 sich fast verdoppelte (siehe Abbildung \ref{fig:uwan}). Die häufigsten Ursachen für einen Notruf von Unverletzten Personen sind {\glqq Verirren und Versteigen\grqq} gefolgt von {\glqq Erschöpfung\grqq}  und {\glqq Wettersturz\grqq}. Es gibt durchaus Faktoren, die nicht vorhersehbar sind, dazu gehören Wettersturz, Blitzschlag, Steinschlag oder Herz-Kreislaufstörungen. Viele Faktoren können jedoch mit richtiger Vorbereitung und Planung durchaus minimiert werden. Recht hohe Zahlen $($zwischen 41 und 105$)$ findet man unter dem Kriterium {\glqq Sonstiges\grqq}. Das bedeutet, dass keines der angegebenen Kriterien für das Absetzen eines Notrufes des Unverletzten zutraf. Hier stellt sich die Frage, ob es sein kann, dass der technische Fortschritt und die Menge an konsumierten Online-Inhalten dazu führt, dass sich Wanderer nach der Planungsphase und während Ausübung der Sportart zu sehr in Sicherheit wiegen und eher für zu schwierige Touren entscheiden?

\begin{figure}
	\centering
	\includegraphics[width=0.7\linewidth]{../../uwan}
	\caption{}
	\label{fig:uwan}
\end{figure}



\section{Zielsetzung und Forschungsfrage}


Das Ziel der Bachelorarbeit ist es zu erforschen, wie weit  Wanderer von neuen Technologien und konsumierten Online-Inhalten bei der Risikobewertung einer Bergtour beeinflusst werden. Dabei soll einerseits untersucht werden, wie gut Wanderer, die ihnen zur Verfügung stehenden Technologien anwenden. Andererseits wird erforscht, wie Wanderer mit Online-Inhalten umgehen und die nützlichen Informationen filtern und mit ihren Fähigkeiten abgleichen. In einer qualitativen Umfrage soll herausgefunden werden, welche Technologien und Online-Inhalte Wanderer bevorzugt nutzen, um sich ein Bild von einer Tour zu machen. Des Weiteren soll herausgefunden werden, welchen Online-Inhalten Wanderer vertrauen und welche Inhalte für die Entscheidungsfindung, welche Tour tatsächlich gegangen wird, herangezogen werden.\\
Daraus lässt sich folgende Forschungsfrage ableiten: 
\textit{Wie werden Wanderer von Online-Inhalten und neuen Technologien bei der Risikobewertung einer Tour beeinflusst?}

\section{Aufbau der Arbeit}

Die vorliegende Bachelorarbeit \textit{Analyse des Entscheidungsverhaltens von Bergsportlern bei der Risikobewertung bei der Auswahl der Route beim Ausüben von Bergsport mit Zuhilfenahme von mobilen Applikationen und Online-Inhalten} besteht aus \# \# \# \# \# \# \# \# Kapiteln.\par

Im ersten Kapitel soll an das Thema der Arbeit herangeführt werden, die Relevanz und Problemstellung werden erläutert und eine Zielsetzung inklusive der Forschungsfrage werden dargestellt.\par

Die Kapitel \# \# \# \# \# \# \# \# enthalten die theoretischen Grundlagen.\par

Im \# \# \# \# \# \# \# \# Kapitel wird die Forschungsmethodik erläutert\par

Im \# \# \# \# \# \# \# \# Kapitel werden die Ergebnisse vorgestellt und die Arbeit schließt mit einem Fazit und Ausblick im \# \# \# \# \# \# \# \# Kapitel ab.

\chapter{Grundlagen}
\label{ch:Grundlagen}

Im folgenden Kapitel werden die grundlegenden Begriffe, welche für die Bachelorarbeit relevant sind, erläutert. 

\section{Web 2.0}
\label{web2.0}

Das Internet wurde in seinen Anfängen für Informations- und Darstellungszwecke benutzt. Dieses Konzept wird als Web 1.0 bezeichnet. Im Web 1.0 gibt es nur wenige Content-Ersteller, die überwiegende Mehrheit der Internetnutzer agiert lediglich als Konsument von Inhalten\footcite{unterschiedWeb1u2}. Darauf folgte Anfang der 2000er Jahre das Konzept des Web 2.0. Der Begriff wurde von O'Reilly geprägt, wobei der Begriff bereits zuvor in Gebrauch war. Web 2.0 beschreibt keine neue Technologie sondern vielmehr eine veränderte Nutzung des Internets. O'Reilly beschrieb das Web 2.0 als eine Reihe von Prinzipien und Praktiken, die wie ein Sonnensystem von Websites miteinander verbunden sind, wobei einige oder alle dieser Prinzipien in unterschiedlichem Abstand vom Kern sind\footcite{oreilly}. O'Reilly sieht das Web 2.0 als eine von allen nutzbare Plattform mit Fokus auf datenbasierte Dienste.
Die Trennung, ob eine Internetseite eine Web 1.0 oder Web 2.0-Seite ist, lässt sich nur bei der Analyse mehrerer Achsen sichtbar machen. Dazu muss man sich die technische, strukturelle und soziologische Achse anschauen. Bei der technischen Achse betrachtet man die Präsentationstechnologie, die sowohl zur Darstellung der Webseite als auch zur Interaktion mit dem Benutzer verwendet wird. An der strukturellen Achse wird der Zweck und das Layout der Seite betrachtet und auf der soziologischen Achse werden die Freunde und Gruppen betrachtet\footcite{unterschiedWeb1u2}. \\
Durch die neu entstandene kollaborative Zusammenarbeit ist es möglich, das Wissen vieler Internetnutzer der Allgemeinheit zugänglich zu machen. Somit haben sich Webseiten im Web 2.0 so verändert, dass sie für die interaktive und kollaborative Nutzung die geeigneten Plattformen zur Verfügung stellen und die Webseiten laufend durch die Nutzerbeteiligung auf dem aktuellen Stand gehalten werden. Durch die Nutzerbeteiligung wurden Webseiten und deren Inhalte dynamischer und flexibler.
Durch neue Technologien und das Verlangen des Internetnutzers, seine Erfahrungen, Kenntnisse und Gedanken mitzuteilen, entstand eine neue Interaktivität.


\subsection{Social Web}

Während das Internet genutzt wurde, um die soziale Interaktion zu erleichtern, ermöglichte die Entstehung und schnelle Verbreitung von Web 2.0-Funktionalitäten im ersten Jahrzehnt des 21. Jahrhunderts einen Evolutionssprung in der sozialen Komponente der Webnutzung\footcite[S. 745-750]{obarSocialMedia}. Das Internet hat sich von einem Abrufmedium zu einem partizipativen Mitmachmedium\footcite{panke} entwickelt. Das Social Web kann als Teilbereich des Web 2.0 gesehen werden. Es entwickelte sich, da immer mehr Menschen Zugang zum Internet hatten und die Bedeutung des Internets zunahm sowie dem sozio-technischen Wandel der Netznutzung\footcite[S. 350]{griesbaum}. Einen erheblichen Teil zur Etablierung brachten die geringen technischen Anforderungen hinsichtlich des Betriebes und der Nutzung von Social Software, dazu gehören Programme und Anwendungen. Die Nutzung von dieser Software wurde leichter, sodass sie nicht mehr nur von Experten bedient werden konnte, dies erleichterte die Erstellung von Inhalten und den sozialen Austausch.
Das Social Web besteht aus folgenden Bereichen:
\begin{itemize}
	\item \textbf{Informationsaustausch}
	\item \textbf{Beziehungsaufbau}
	\item \textbf{Kollaborative Zusammenarbeit}
	\item \textbf{Kommunikation}
\end{itemize}


\subsection{Social Media}
\label{sozialMedia}


Für den Begriff Social Media gibt es viele Definitionen. Andreas M. Kaplan und Michael Haenlein beschreiben den Begriff Social Media als eine Gruppe von Internet-basierten Anwendungen, die auf den ideologischen und technologischen Grundlagen des Web 2.0 aufbauen und die Erstellung und den Austausch von User Generated Content ermöglichen\footcite[S. 59-68]{kaplan}.\\
Kietzmann et al. verwenden eine Social Media-Bienenwabe mit sieben funktionalen Bausteinen, welche die verschiedenen Ebenen von Social Media abbilden\footcite[S.241-251]{kietzmann}:

\begin{itemize}
	\item \textbf{Identität:} Offenbarung der Identität des Nutzers. Nach Kaplan und Haenlein ist die Identität eines Nutzer oft durch die bewusste oder unbewusste Selbstdarstellung subjektiver Informationen wie Gedanken, Gefühle, Vorlieben und Abneigungen gekoppelt\footcite[S.59-68]{kaplan}.
	
	\item \textbf{Gespräche:} dienen dem Nutzer um mit anderen Nutzern in einem Social Media Umfeld zu kommunizieren um beispielsweise Gleichgesinnte zu finden oder ihre Botschaft zu Gehör zu bringen.
	
	\item \textbf{Teilen:} Benutzer ändern, verteilen und empfangen Inhalte, darunter fallen z.B.: Text, Bild, Video, Ton, Link und Ort. Das Teilen ermöglicht Benutzern persönliche Objekte - Erfahrungen und Beobachtungen - mit der Welt zu teilen.
	
	
	\item \textbf{Präsenz:} Benutzer sehen, ob andere Benutzer erreichbar sind, dies soll eine Brücke zwischen der realen und der virtuellen Welt darstellen.
	
	
	\item \textbf{Beziehungen:} Beziehungen zeigen, wie Benutzer mit anderen Benutzern in Beziehung gesetzt werden können. Beziehungen sind von Plattform zu Plattform unterschiedlich wichtig und geregelt. In Blogs zum Beispiel können Benutzer eine Beziehung zueinander aufbauen, ohne eine formale Vereinbarung darüber zu treffen, welche Informationen sie teilen.
	
	
	\item \textbf{Reputation:} Reputation ist das Ausmaß, in welchem die Nutzer das Ansehen anderer, auch ihrer selbst, in einem Social Media-Umfeld, erkennen können. In Social Media bezieht sich die Reputation aber nicht nur auf Personen sondern auch auf Inhalte. In den meisten Fällen basiert Reputation auf Vertrauen, da Informationstechnologie nicht in der Lage ist, solche qualitativen Kriterien zu bestimmen. Inhalte können mit Content-Voting-Systemen bewertet werden, dies können beispielsweise {\glqq likes\grqq}, {\glqq Daumen hoch\grqq}, {\glqq Daumen runter\grqq} aber auch Anzahl der Views oder Anzahl der Follower sein.
	
	
	\item \textbf{Gruppen:} Benutzer können Gemeinschaften und Subgemeinschaften bilden. Es gibt zwei Arten von Gemeinschaften, entweder sortieren Einzelpersonen ihre Kontakte selbst oder sie bilden Gruppen analog zu Clubs. Diese Gruppen können für jedermann offen, geschlossen oder geheim sein.
\end{itemize}

HIER KOMMT NOCH EIN ÜBERGANG


\section{User Generated Content}
\label{ch:ugc}













\section{Angewandte Modelle}
\label{ch:Modelle}

In diesem Teil wird auf die beiden gängigsten Modelle, dass Technology Acceptance Model von Fred D. Davis 

\subsection{Technology Acceptance Model}
\label{TAM}

Der wesentliche Faktor dieses Modells ist die Technikakzptanz \footcite[S. 319]{Davis}

\subsection{Task -Technology-Fit-Model}
\label{ch:TTF}

%%EOF@fat




 	% Hauptteil

%	\input{content/beispiele}

%============================================================
% Abspann mit Anhängen und Quellenverzeichnis
%============================================================
\appendix 	% Ab hier Nummerierung A, B, ...
\clearpage
\onehalfspacing
	%% 	This LaTeX Template is meant for "Second Bachelor´s Thesis"
%% 	written by Frank A Titze in March 2009
%% 	for Hochschule Kufstein Tirol, Austria, Dep. Wirtschaftsinformatik
%% 	www.hsk-edu.at
%% 	@author Frank A Titze (frank.a.titze@gmx.eu)
%% 	@copyright 2009, FAT
%% 	@license GPL
%% 	@version 1.0
%%=================================================
%% 	Main Document  Vorlage_BA2_by.fat.tex
%% 	This document:  appendix.tex
%%=================================================

\chapter{Ein Anhang}



\chapter{Noch ein Anhang}

%%EOF@fat
 	% Anhänge

\newpage
\singlespacing
	\nocite{*} 	% NUR FÜR TESTZWECKE!!! Damit provoziert man ein vollständiges Literaturverzeichnis ALLER bib-Einträge; normalerweise werden nur die im Text wirklich benutzten Quellen ins Verzeichnis aufgenommen
	\bibliographystyle{natdin} 	% Literaturverzeichnis nach DIN
	\bibliography{misc/myStandardBIB} 	% Datei der Jabref Bibliotheksdatei .bib
	
%============================================================
\end{document}

%%EOF@fat
